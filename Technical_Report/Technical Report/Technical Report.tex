% --------------------------------------------------------------
% This is all preamble stuff that you don't have to worry about.
% Head down to where it says "Start here"
% --------------------------------------------------------------
 
\documentclass[10pt]{article}
%\setlength\parindent{24pt}  ??????????

\usepackage[margin=0.8in]{geometry} 
\usepackage{amsmath,amsthm,amssymb}
\usepackage{setspace}
\usepackage{amssymb}
\usepackage{amsmath}
\usepackage{amsthm}
\usepackage{latexsym}
\usepackage{enumerate}
\usepackage{setspace}
\usepackage{mathrsfs}
\usepackage[colorlinks=true, urlcolor=blue]{hyperref} % This package enables hyperlinks
\renewcommand{\qedsymbol}{$\blacksquare$} %This command makes the end-of-proof boxes solid black.
\usepackage{amscd}
\usepackage{mathtools}
\usepackage{graphicx}

\usepackage{siunitx} % Required for alignment

\sisetup{
  round-mode          = places, % Rounds numbers
  round-precision     = 2, % to 2 places
}


\makeatletter
\renewcommand*\env@matrix[1][*\c@MaxMatrixCols c]{%controls position of matrix elements.
  \hskip -\arraycolsep
  \let\@ifnextchar\new@ifnextchar
  \array{#1}}
\makeatother
 
\newcommand{\N}{\mathbb{N}}
\newcommand{\Z}{\mathbb{Z}}
\def\ep{\varepsilon}
\def\bR{\mathbb{R}}
\def\S{\mathbb{S}}
\def\C{\mathbb{C}}
\def\CP{\mathbb{CP}}
\def\bA{\mathscr{A}}
\def\lA{\lambda}
\def\a{\alpha}
\def\g{\gamma}
\def\e{\epsilon}
\def\p{\partial}
\def\sB{\mathscr{B}}
\def\la{\langle}
\def\ra{\rangle}
\newcommand{\norm}[1]{\left\lVert#1\right\rVert}
\def\n{\norm}
\def\o{\overline}


\newmuskip\pFqskip % this code is for the Hypergeometric notation
\pFqskip=6mu
\mathchardef\pFcomma=\mathcode`, % keep a copy of the comma

\newcommand*\pQq[5]{%this code is a continuation of the Hypergeometric notation
  \begingroup
  \begingroup\lccode`~=`,
    \lowercase{\endgroup\def~}{\pFcomma\mkern\pFqskip}%
  \mathcode`,=\string"8000
  {}_{#1}\phi_{#2}\biggl(\genfrac..{0pt}{}{#3}{#4};#5\biggr)%
  \endgroup
}

\renewcommand{\labelitemi}{$\textendash$} %this code resets the top list marker to a dash

 
\newenvironment{theorem}[2][Theorem]{\begin{trivlist}
\item[\hskip \labelsep {\bfseries #1}\hskip \labelsep {\bfseries #2.}]}{\end{trivlist}}
\newenvironment{lemma}[2][Lemma]{\begin{trivlist}
\item[\hskip \labelsep {\bfseries #1}\hskip \labelsep {\bfseries #2.}]}{\end{trivlist}}
\newenvironment{exercise}[2][Exercise]{\begin{trivlist}
\item[\hskip \labelsep {\bfseries #1}\hskip \labelsep {\bfseries #2.}]}{\end{trivlist}}
\newenvironment{reflection}[2][Reflection]{\begin{trivlist}
\item[\hskip \labelsep {\bfseries #1}\hskip \labelsep {\bfseries #2.}]}{\end{trivlist}}
\newenvironment{proposition}[2][Proposition]{\begin{trivlist}
\item[\hskip \labelsep {\bfseries #1}\hskip \labelsep {\bfseries #2.}]}{\end{trivlist}}
\newenvironment{corollary}[2][Corollary]{\begin{trivlist}
\item[\hskip \labelsep {\bfseries #1}\hskip \labelsep {\bfseries #2.}]}{\end{trivlist}}


% For an exam, single spacing is most appropriate
%\singlespacing
% \onehalfspacing%%%
% \doublespacing

% For an exam, we generally want to turn off paragraph indentation
\parindent 0ex
%\def\baselinestretch{2} % double spacing

\begin{document}
%\doublespacing
 
% --------------------------------------------------------------
%                         Start here
% --------------------------------------------------------------
 
%\renewcommand{\qedsymbol}{\filledbox}
 
\title{Lake Michigan Influences}%replace X with the appropriate number
\author{Blake Wallace\\ %replace with your name
Capstone Technical Report} %if necessary, replace with your course title
\date{May 17, 2019}
 
\maketitle

\vspace{2mm}

\begin{enumerate}[\null]

\item \textbf{Data science objectives}:
\begin{enumerate}
\item[1.] Are there any statistically significant differences in temperature closer to the water?
\item[2.] Can a predictive model that explains at least 80\% of the variance in the precipitation differences be constructed?
\end{enumerate}
 


\item \textbf{Data Sources}: 

\begin{figure}[!h!]
  \includegraphics[width=0.8\linewidth]{./photos/garden_ohare_different_colors.png}
  \centering
  \caption{In the top right is the location of the weather tower inside of the Chicago Botanical Gardens, while the bottom left shows the location of the tower in the O'Hare Airport.}
  \label{fig:garden_ohare}
\end{figure}

\begin{enumerate}
\item[] \href{https://www.ncdc.noaa.gov/cdo-web/datasets/GHCND/stations/GHCND:USW00094846/detail}{\textbf{CHICAGO OHARE INTERNATIONAL AIRPORT, IL US}}\newline
 - Source: \href{https://www.ncdc.noaa.gov/cdo-web/search}{National Centers for Environmental Information}\newline
 - \href{https://www1.ncdc.noaa.gov/pub/data/cdo/documentation/GHCND_documentation.pdf}{GHCN (Global Historical Climatology Network) Daily Documentation}\newline
  - ID	GHCND:USW00094846  \newline
  - 41.995 N 87.9336 W \newline
 - \href{https://www.flychicago.com/ohare/home/pages/default.aspx}{Airport Information}
 
\begin{figure}[h!]
  \includegraphics[width=0.5\textwidth]{./photos/ohare.png}
  \centering
  \caption{In the top left is the location of the weather tower inside of O'Hare Airport.}
  \label{fig:ohare}
\end{figure}

\item[] \href{https://www.ncdc.noaa.gov/cdo-web/datasets/GHCND/stations/GHCND:USC00111497/detail}{\textbf{CHICAGO BOTANIC GARDEN, IL US}}\newline
 - Source: \href{https://www.ncdc.noaa.gov/cdo-web/search}{National Centers for Environmental Information}\newline
 - \href{https://www1.ncdc.noaa.gov/pub/data/cdo/documentation/GHCND_documentation.pdf}{GHCN (Global Historical Climatology Network) ? Daily Documentation}\newline
  - ID GHCND:USC00111497\newline
  - 42.13987 N 87.78537 W\newline
 - \href{https://www.chicagobotanic.org/?gclid=CjwKCAjw8e7mBRBsEiwAPVxxiFAzbi0I4VKZUO1z3uxcDI36xORzYwbOtmUWGVoUqxRHEi8elJFV2RoCmaoQAvD_BwE}{Garden Information}
 
\begin{figure}[h!]
  \includegraphics[width=0.5\textwidth]{./photos/garden.png}
  \centering
  \caption{The weather tower at the Chicago Botanical Gardens.}
  \label{fig:garden}
\end{figure}

 
\item[] \href{https://www.google.com/search?q=lake+michigan&oq=lake+michigan&aqs=chrome.0.69i59j69i60l3j69i59j0.3015j0j9&sourceid=chrome&ie=UTF-8}{\textbf{Lake Michigan}}
\newline
 - Source: \href{https://coastwatch.glerl.noaa.gov/statistic/statistic.html}{Great Lakes Statistics: Average Surface Water Temperature from the Great Lakes Surface Environmental Analysis (GLSEA)}\newline
 - 44.0 -87.0 (44� 00' 0.00" N 87� 00' 0.00" W)\newline
 - \href{https://coastwatch.glerl.noaa.gov/ftp/glsea/avgtemps/2018/glsea-temps2018_1024.dat}{Data Set for 2018}

\item[] \href{https://www.ndbc.noaa.gov/station_page.php?station=fsti2}{\textbf{Station FSTI2 - Foster Ave., Chicago, IL}} \newline
 - Source: \href{https://www.ndbc.noaa.gov/}{National Data Buoy Center} \newline
 - Owned and maintained by \href{https://www.ndbc.noaa.gov/ndbcexit.php?url=https://wqdatalive.com/public/16&blurb=Chicago+Park+District}{Chicago Park District}\newline
  - 41.976 N 87.648 W (41�58'35" N 87�38'51" W)
\end{enumerate}

\begin{figure}[h!]
  \includegraphics[width=0.5\textwidth]{./photos/FSTI2.png}
  \centering
  \caption{In the top left is the location of the weather tower inside of O'Hare Airport.}
  \label{fig:buoy}
\end{figure}


\textbf{Data}
\begin{enumerate}
\item[] \text{Links}
	\begin{enumerate}
\item[] \href{https://render.githubusercontent.com/view/ipynb?commit=3611d432f79b839ababa4ab30f3e6c71b4c65107&enc_url=68747470733a2f2f7261772e67697468756275736572636f6e74656e742e636f6d2f426c616b6557616c6c6163652f4c616b655f4d6963686967616e5f496e666c75656e6365732f333631316434333266373962383339616261626134616233306633653663373162346336353130372f646174612f446174615f44696374696f6e61726965732f446174615f44696374696f6e6172792e6970796e62&nwo=BlakeWallace%2FLake_Michigan_Influences&path=data%2FData_Dictionaries%2FData_Dictionary.ipynb&repository_id=183659877&repository_type=Repository#Data-Dictionary}{\textbf{Data Dictionary}}
\item[] \href{https://www.ndbc.noaa.gov/measdes.shtml}{\textbf{Bouy Data Dictionary}}
\item[] \href{https://www1.ncdc.noaa.gov/pub/data/ghcn/daily/readme.txt}{\textbf{O'Hare Airport Data Dictionary}}
\item[] \href{https://www1.ncdc.noaa.gov/pub/data/ghcn/daily/readme.txt}{\textbf{Botanical Garden Data Dictionary}}

\end{enumerate}

\item[] \text{"The five core values are:"}
\begin{enumerate}
 \item[] \textbf{ohare\_prcp} - Precipitation (PRCP) (inches)
 \item[] \textbf{ohare\_snfall} - Snowfall (SNOW) (inches)
 \item[] \textbf{ohare\_sndpth} - Snow depth (SNWD) (inches) 
 \item[] \textbf{ohare\_maxtmp} - Maximum temperature (TMAX) (Fahrenheit)
 \item[] \textbf{ohare\_mintmp} - Minimum temperature (TMIN) (Fahrenheit)
\end{enumerate}


\item[] \text{Other Features}
\begin{enumerate}
\item[] \textbf{lake-temp} - Average Daily Surface Water Temperature for Lake Michigan 	(Fahrenheit) 
 \item[] \textbf{garden\_prcp} - Precipitation (PRCP) (inches)
 \item[] \textbf{garden\_maxtmp} - Maximum temperature (TMAX) (Fahrenheit)
 \item[] \textbf{garden\_mintmp} - Minimum temperature (TMIN) (Fahrenheit)
 \item[] \textbf{garden\_tobs} - Temperature at time of observation (TOBS) (Fahrenheit)
 \item[] \textbf{ohare\_wspd} - Average daily wind speed (AWND) (miles per hour) 
 \item[] \textbf{ohare\_atmp} - Average Temperature (TAVG) (Fahrenheit)
 \item[] \textbf{ohare\_w2dir} - Direction of fastest 2-minute wind (WDF2) (the direction the wind is coming from in degrees clockwise from true N)  
 \item[] \textbf{ohare\_w2spd} - Fastest 2-minute wind speed (WSF2) (miles per hour)
\end{enumerate}

\item[] \text{Feature Engineering}
\begin{enumerate}
 \item[] \textbf{target} - absolute difference between the precipitation measurements at Ohare and the garden ( ohare\_prcp - garden\_prcp )
 \item[] \textbf{garden\_didrain} - categorical, 1 for yes, 0 for no
 \item[] \textbf{ohare\_didrain} - categorical, 1 for yes, 0 for no
 \item[] \textbf{garden\_medtmp} - Median daily temperature at the Garden/ midpoint between the max and min temperatures ( (garden\_maxtmp + garden\_mintmp)/2 )
 \item[] \textbf{ohare\_medtmp} - Median daily temperature at ohare/ midpoint between the max and min temperatures ( (ohare\_maxtmp + ohare\_mintmp)/2 )
 \item[] \textbf{tmpdiff} - difference between the median temperatures at ohare and the garden ( ohare\_medtmp - garden\_medtmp )
 \end{enumerate}  
 \end{enumerate}
 
  







\item \textbf{Data Cleaning/Data Manipulation/EDA}: 







\item \textbf{Tests and Evaluation}:

In this section we will perform several two-tailed hypothesis tests, all of the form:

\begin{align*}
 H_0: & \text{ the average temperatures are the same}, \\
 H_A: & \text{ the average temperatures are different}.
\end{align*}


\begin{table}[h!]
	\begin{center}
	\begin{tabular}{|c|c|c|c|c|c|c|}\hline
			\textbf{Data}  & \textbf{Quantity of Data} & \textbf{t-score} & \textbf{p-value} & 				\textbf{Significance} & \textbf{Gardens Avg (F)} & \textbf{Ohare Avg (F)} \\ 				\hline
			All Data    & 7923 & 0.5876  & 0.5568  & None  & 59.24  & 59.43 \\ \hline
			No Rain     & 4022 & 3.285   & 0.0010  & Yes  & 58.99  & 60.57   \\ \hline
			Both Rain   & 1648 & -2.629  & 0.0086  & Yes   & 59.48    & 57.7   \\ \hline
			ohareRain   & 1193 & -1.9557 & 0.0506  & None   & 59.06   & 57.43   \\ \hline
			gardensRain & 1060 & 0.0904  & 0.9280  & None   & 59.99   & 60.07  \\ \hline
	\end{tabular}
        \caption{Statistical Tests with Results}
        \label{table:tests}
        \end{center}
\end{table}

Table \ref{table:tests} shows the results from five different tests.  The first row considers the dataset containing all of the weather measurements from January, 1995 through December, 2018.  The second row represents when there is no rain at either location, the third row when there is rain at both locations, and the last two indicate when there is rain at only the airport, or only the gardens,  respectively.  As can be seen, with a very small $p$-value, there is a significant difference between the average maximum temperatures when it rains at both locations and when it is not raining at either location.  This indicates that, given that there is no precipitation at either location, there is less than 0.1\% chance of rejecting the Null Hypothesis incorrectly.  Since this number is so small, there is evidence to reject the null hypothesis, indicating that there is most likely a difference in the average maximum temperatures between the two locations.  

\begin{figure}[h!]
  \caption{\textbf{Temperature/ Full Data (F)}}
  \label{fig:Ohare_full_dataset}
  \includegraphics[width=0.5\textwidth]{./photos/temp_scatter_full_noTitle.png}
  \centering
\end{figure}

\begin{figure}[h!]
  \includegraphics[width=0.5\textwidth]{./photos/t-test_for_the_full_dataset.png}
  \centering
  \caption{t-statistic for the full dataset}
  \label{fig:t-statisticFull}
\end{figure}

\newpage
\item \textbf{Models and Evaluation}:

\begin{table}[h!]
	\begin{center}
	\caption{Predictive Models with their scores}
	\label{table:models}
	\begin{tabular}{|c|c|c|c|c|c|}
			\hline
				\textbf{Model}   & \textbf{Training score*} & \textbf{Testing score*} & 						\textbf{Training MSE**} & \textbf{Testing MSE**} & \textbf{Cross 					Validation} \\ \hline
				Linear no poly                          & 0.0825         & 0.1052        & 0.0933       					& 0.0683      & 0.0785           \\ \hline
				Linear gs                               & 0.1222         & 0.1329        & 0.0893       & 					0.0662      & 0.0984           \\ \hline
				Decision Tree                           & 0.1139         & 0.0691        & 0.0901       					& 0.0711      & 0.0429           \\ \hline
				Decision Tree gs                        & 0.0937         & 0.0584        & 0.0922       					& 0.0719      & 0.0450           \\ \hline
				Random Forest                           & 0.8614         & 0.0517        & 0.0134       					& 0.0724      & 0.0554           \\ \hline
				Random Forest                           & 0.8711         & 0.1078        & 0.0131       					& 0.0681      & 0.0770           \\ \hline
				Random Forest gs                        & 0.8658         & 0.0905        & 0.0136       					& 0.0694      & 0.0651           \\ \hline
				Random Forest                           & 0.8677         & 0.0682        & 0.0135       					& 0.0711      & 0.0660           \\ \hline
				Random Forest                           & 0.8704         & 0.1153        & 0.0132       					& 0.0676      & 0.0767           \\ \hline
				Random Forest                           & 0.8080         & 0.0957        & 0.0195       					& 0.0690      & 0.0787           \\ \hline
				Random Forest ada                       & 0.9547         & 0.0549        & 0.0331       					& 0.0722      & 0.0331           \\ \hline
				Random Forest ada                       & 0.9445         & 0.0525        & 0.0056       					& 0.0723      & 0.0283           \\ \hline
				\multicolumn{1}{|l|}{Random Forest bag} & 0.6735         & 0.1130        & 					0.0332       & 0.0677      & 0.0928           \\ \hline
				\multicolumn{1}{|l|}{Random Forest bag} & 0.6705         & 0.1239        & 					0.0335       & 0.0669      & 0.0943           \\ \hline
	\end{tabular}
	\end{center}
\textbf{*} The score refers to the Coefficient of Determination. 

\textbf{**} MSE - Mean Squared Error

\textbf{gs} denotes a Grid Search was performed.

\textbf{ada} denotes an Ada Boost model was performed.
\end{table}

\newpage
\item \textbf{Future Iterations}:

The presence of the statistically significant differences between the average maximum daily temperatures closer to the water and further away from the water is striking, and this must be the next line of investigation. It is likely that there are seasons of the year when more rain falls, traditionally in the Spring, when the Lake Michigan water temperature is usually close to it's lowest point.  It is believed that, since the temperature difference when there is no rain has a lower temperature at the Botanical Gardens, that this must occur more often when the lake is at its lowest temperature and climbing.  As depicted in Figure \ref{fig:lake_temp_nov2016_jun2017}, the lowest point of the lake temperatures occurs around the end of winter, beginning of spring.  This will be the next line of investigation.

Regarding modeling, the next step will be to investigate temperature fields around the Chicagoland area.  To do this, it is necessary to consider data from many other locations.  While the problem of incomplete data will be present, increasing the volume, and time-correlating it will mean that a less lengthy time period will be necessary to capture the temperature trends.  The predictive element is present when extrapolating temperature in the holes between the specific locations.  The spatio-temporal nature of this problem means a more sophisticated approach must be taken to predict temperatures in the gaps between locations where temperature measurements are made. 

\begin{figure}[h!]
  \caption{\textbf{Lake Michigan Avgs, Nov 2016 - June 2017}}
  \label{fig:lake_temp_nov2016_jun2017}
  \includegraphics[width=0.5\textwidth]{./photos/lake_temp_nov2016_jun2017.png}
  \centering
\end{figure}
\item \textbf{Resources}:



\end{enumerate}

% --------------------------------------------------------------
%     You don't have to mess with anything below this line.
% --------------------------------------------------------------
 
\end{document}