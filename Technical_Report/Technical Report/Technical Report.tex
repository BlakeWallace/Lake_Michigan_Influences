% --------------------------------------------------------------
% This is all preamble stuff that you don't have to worry about.
% Head down to where it says "Start here"
% --------------------------------------------------------------
 
\documentclass[10pt]{article}
%\setlength\parindent{24pt}  ??????????

\usepackage[margin=0.8in]{geometry} 
\usepackage{amsmath,amsthm,amssymb}
\usepackage{setspace}
\usepackage{amssymb}
\usepackage{amsmath}
\usepackage{amsthm}
\usepackage{latexsym}
\usepackage{enumerate}
\usepackage{setspace}
\usepackage{mathrsfs}
\usepackage{hyperref} % This package enables hyperlinks
\renewcommand{\qedsymbol}{$\blacksquare$} %This command makes the end-of-proof boxes solid black.
\usepackage{amscd}
\usepackage{mathtools}


\makeatletter
\renewcommand*\env@matrix[1][*\c@MaxMatrixCols c]{%controls position of matrix elements.
  \hskip -\arraycolsep
  \let\@ifnextchar\new@ifnextchar
  \array{#1}}
\makeatother
 
\newcommand{\N}{\mathbb{N}}
\newcommand{\Z}{\mathbb{Z}}
\def\ep{\varepsilon}
\def\bR{\mathbb{R}}
\def\S{\mathbb{S}}
\def\C{\mathbb{C}}
\def\CP{\mathbb{CP}}
\def\bA{\mathscr{A}}
\def\lA{\lambda}
\def\a{\alpha}
\def\g{\gamma}
\def\e{\epsilon}
\def\p{\partial}
\def\sB{\mathscr{B}}
\def\la{\langle}
\def\ra{\rangle}
\newcommand{\norm}[1]{\left\lVert#1\right\rVert}
\def\n{\norm}
\def\o{\overline}


\newmuskip\pFqskip % this code is for the Hypergeometric notation
\pFqskip=6mu
\mathchardef\pFcomma=\mathcode`, % keep a copy of the comma

\newcommand*\pQq[5]{%this code is a continuation of the Hypergeometric notation
  \begingroup
  \begingroup\lccode`~=`,
    \lowercase{\endgroup\def~}{\pFcomma\mkern\pFqskip}%
  \mathcode`,=\string"8000
  {}_{#1}\phi_{#2}\biggl(\genfrac..{0pt}{}{#3}{#4};#5\biggr)%
  \endgroup
}

\renewcommand{\labelitemi}{$\textendash$} %this code resets the top list marker to a dash

 
\newenvironment{theorem}[2][Theorem]{\begin{trivlist}
\item[\hskip \labelsep {\bfseries #1}\hskip \labelsep {\bfseries #2.}]}{\end{trivlist}}
\newenvironment{lemma}[2][Lemma]{\begin{trivlist}
\item[\hskip \labelsep {\bfseries #1}\hskip \labelsep {\bfseries #2.}]}{\end{trivlist}}
\newenvironment{exercise}[2][Exercise]{\begin{trivlist}
\item[\hskip \labelsep {\bfseries #1}\hskip \labelsep {\bfseries #2.}]}{\end{trivlist}}
\newenvironment{reflection}[2][Reflection]{\begin{trivlist}
\item[\hskip \labelsep {\bfseries #1}\hskip \labelsep {\bfseries #2.}]}{\end{trivlist}}
\newenvironment{proposition}[2][Proposition]{\begin{trivlist}
\item[\hskip \labelsep {\bfseries #1}\hskip \labelsep {\bfseries #2.}]}{\end{trivlist}}
\newenvironment{corollary}[2][Corollary]{\begin{trivlist}
\item[\hskip \labelsep {\bfseries #1}\hskip \labelsep {\bfseries #2.}]}{\end{trivlist}}


% For an exam, single spacing is most appropriate
%\singlespacing
% \onehalfspacing%%%
% \doublespacing

% For an exam, we generally want to turn off paragraph indentation
\parindent 0ex
%\def\baselinestretch{2} % double spacing

\begin{document}
%\doublespacing
 
% --------------------------------------------------------------
%                         Start here
% --------------------------------------------------------------
 
%\renewcommand{\qedsymbol}{\filledbox}
 
\title{Lake Michigan Influences}%replace X with the appropriate number
\author{Blake Wallace\\ %replace with your name
Capstone Technical Report} %if necessary, replace with your course title
\date{May 14, 2019}
 
\maketitle

\vspace{2mm}

\begin{enumerate}[\null]

\item \textbf{Data science objectives}:
\begin{enumerate}
\item[i.] Does Rain have an effect on the average daily water temperature?
\item[ii.] What effect does rain have on the average daily temperature near the water?
\item[iii.] What effect does rain have on the average daily temperature far from the water?
\item[iv.] When it rains, is there a statistically significant difference between the amount of rain that falls in downtown Chicago compared to the Ohare airport?
\item[v.] How much correlation exists between the average daily temperature of Lake Michigan and the temperature difference between the downtown Chicago area and the Ohare airport?
\item[vi.] Can we build a model that predicts with at least 80\% accuracy the difference in total precipitation between Ohare airport and the the Botanical gardens?
\item[vii.] Is there any statistically significant difference between the daily temperature near the water as apposed to far from the water?
\end{enumerate}
 


\item \textbf{Data}: 

\text{Links}
\begin{enumerate}
\item[] \href{http://localhost:8888/notebooks/Projects/Repos_BlakeWallace/Capstone/working/EDA_and_analysis.ipynb#Data-Dictionary}\textbf{Data Dictionary}
\item[] \href{https://www.ndbc.noaa.gov/measdes.shtml} \textbf{Bouy Data Dictionary}
\item[] \href{https://www1.ncdc.noaa.gov/pub/data/ghcn/daily/readme.txt} \textbf{O'Hare Airport Data Dictionary}
\item[] \href{https://www1.ncdc.noaa.gov/pub/data/ghcn/daily/readme.txt} \textbf{Botanical Garden Data Dictionary}

\end{enumerate}

\text{"The five core values are:"}
\begin{enumerate}
 \item[] \textbf{ohare\_prcp} - Precipitation (PRCP) (inches)
 \item[] \textbf{ohare\_snfall} - Snowfall (SNOW) (inches)
 \item[] \textbf{ohare\_sndpth} - Snow depth (SNWD) (inches) 
 \item[] \textbf{ohare\_maxtmp} - Maximum temperature (TMAX) (Fahrenheit)
 \item[] \textbf{ohare\_mintmp} - Minimum temperature (TMIN) (Fahrenheit)
\end{enumerate}

\text{Other Features}
\begin{enumerate}
\item[] \textbf{lake-temp} - Average Daily Surface Water Temperature for Lake Michigan 	(Fahrenheit) 
 \item[] \textbf{garden\_prcp} - Precipitation (PRCP) (inches)
 \item[] \textbf{garden\_maxtmp} - Maximum temperature (TMAX) (Fahrenheit)
 \item[] \textbf{garden\_mintmp} - Minimum temperature (TMIN) (Fahrenheit)
 \item[] \textbf{garden\_tobs} - Temperature at time of observation (TOBS) (Fahrenheit)
 \item[] \textbf{ohare\_wspd} - Average daily wind speed (AWND) (miles per hour) 
 \item[] \textbf{ohare\_atmp} - Average Temperature (TAVG) (Fahrenheit)
 \item[] \textbf{ohare\_w2dir} - Direction of fastest 2-minute wind (WDF2) (the direction the wind is coming from in degrees clockwise from true N)  
 \item[] \textbf{ohare\_w2spd} - Fastest 2-minute wind speed (WSF2) (miles per hour)
\end{enumerate}

\text{Feature Engineering}
\begin{enumerate}
 \item[] \textbf{target} - absolute difference between the precipitation measurements at Ohare and the garden ( ohare\_prcp - garden\_prcp )
 \item[] \textbf{garden\_didrain} - categorical, 1 for yes, 0 for no
 \item[] \textbf{ohare\_didrain} - categorical, 1 for yes, 0 for no
 \item[] \textbf{garden\_medtmp} - Median daily temperature at the Garden/ midpoint between the max and min temperatures ( (garden\_maxtmp + garden\_mintmp)/2 )
 \item[] \textbf{ohare\_medtmp} - Median daily temperature at ohare/ midpoint between the max and min temperatures ( (ohare\_maxtmp + ohare\_mintmp)/2 )
 \item[] \textbf{tmpdiff} - difference between the median temperatures at ohare and the garden ( ohare\_medtmp - garden\_medtmp )
 \end{enumerate}  
 
  







\item \textbf{Data Cleaning/Data Manipulation/EDA}: 



\item \textbf{Models and Evaluation}:

\begin{table}[h]
\begin{tabular}{|c|c|c|c|c|c|}
\hline
Model                                   & Training score & Testing score & Training MSE & Testing MSE & cross validation \\ \hline
Linear no poly                          & 0.0825         & 0.1052        & 0.0933       & 0.0683      & 0.0785           \\ \hline
Linear gs                               & 0.1222         & 0.1329        & 0.0893       & 0.0662      & 0.0984           \\ \hline
Decision Tree                           & 0.1139         & 0.0691        & 0.0901       & 0.0711      & 0.0429           \\ \hline
Decision Tree gs                        & 0.0937         & 0.0584        & 0.0922       & 0.0719      & 0.0450           \\ \hline
Random Forest                           & 0.8614         & 0.0517        & 0.0134       & 0.0724      & 0.0554           \\ \hline
Random Forest                           & 0.8711         & 0.1078        & 0.0131       & 0.0681      & 0.0770           \\ \hline
Random Forest gs                        & 0.8658         & 0.0905        & 0.0136       & 0.0694      & 0.0651           \\ \hline
Random Forest                           & 0.8677         & 0.0682        & 0.0135       & 0.0711      & 0.0660           \\ \hline
Random Forest                           & 0.8704         & 0.1153        & 0.0132       & 0.0676      & 0.0767           \\ \hline
Random Forest                           & 0.8080         & 0.0957        & 0.0195       & 0.0690      & 0.0787           \\ \hline
Random Forest ada                       & 0.9547         & 0.0549        & 0.0331       & 0.0722      & 0.0331           \\ \hline
Random Forest ada                       & 0.9445         & 0.0525        & 0.0056       & 0.0723      & 0.0283           \\ \hline
\multicolumn{1}{|l|}{Random Forest bag} & 0.6735         & 0.1130        & 0.0332       & 0.0677      & 0.0928           \\ \hline
\multicolumn{1}{|l|}{Random Forest bag} & 0.6705         & 0.1239        & 0.0335       & 0.0669      & 0.0943           \\ \hline
\end{tabular}
\end{table}

\item \textbf{Future Iterations}:

\item \textbf{Resources}:

1.  An executive summary:
What is your goal?
Where did you get your data?
What are your metrics?
What were your findings?
What risks/limitations/assumptions affect these findings?
2. Summarize your statistical analysis, including:
implementation
evaluation
inference
3. Clearly document and label each section of your notebook(s)
Logically organize your information in a persuasive, informative manner.
Include notebook headers and subheaders, as well as clearly formatted markdown for all written components.
Include graphs/plots/visualizations with clear labels.
Comment and explain the purpose of each major section/subsection of your code.
Document your code for your future self, as if another person needed to replicate your approach.
4. Clearly document all of your decision points in the relevant sections
How did you acquire your data?
How did you transform or engineer your data? Why?
How did you select your model?
How did you optimize hyperparameters?
5. Host your notebook and any other materials in your own public Github Repository.
You repo should have README file that guides us through the repository and links to important files.
Include links and explanations to any outside libraries or source code used.
Host a copy of your dataset or include a link to a remotely hosted version.

\end{enumerate}

% --------------------------------------------------------------
%     You don't have to mess with anything below this line.
% --------------------------------------------------------------
 
\end{document}